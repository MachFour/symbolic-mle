\documentclass[11pt,paper=a4,numbers=noendperiod]{scrartcl} % headings=small,
\KOMAoptions{DIV=12}

\usepackage{amsmath}
\usepackage{amsfonts}
\usepackage{amssymb}

\usepackage[utf8]{inputenc}

\usepackage{hyperref}
\hypersetup{colorlinks=true}

% only number down to 'part' (i.e. don't number sections or below)
\setcounter{secnumdepth}{\partnumdepth}

%\addtolength{\topmargin}{-4ex}
\addtolength{\textheight}{8ex}

% disable paragraph indent
\setlength{\parindent}{0pt}
\usepackage{parskip}

\newcommand{\R}{\mathbb{R}}
\newcommand{\Q}{\mathbb{Q}}
\newcommand{\C}{\mathbb{C}}
\newcommand{\N}{\mathcal{N}}
\newcommand{\NN}{\mathbb{N}}
\newcommand{\F}{\mathbb{F}}
\newcommand{\PP}{\mathbb{P}}
\newcommand{\T}{\mathbb{T}}
\newcommand{\Z}{\mathbb{Z}}
\newcommand{\B}{\mathfrak{B}}
\newcommand{\BB}{\mathcal{B}}
\newcommand{\M}{\mathfrak{M}}
\newcommand{\X}{\mathfrak{X}}
\newcommand{\Y}{\mathfrak{Y}}
\newcommand{\CC}{\mathcal{C}}
\newcommand{\E}{\mathbb{E}}
\newcommand{\cP}{\mathcal{P}}
\newcommand{\cS}{\mathcal{S}}
\newcommand{\A}{\mathcal{A}}
\newcommand{\ZZ}{\mathcal{Z}}


%%%%%%%%%%%%%%%%%%%%%%%
%% Max's definitions %%
%%%%%%%%%%%%%%%%%%%%%%%

\newcommand{\su}{\ensuremath{s_{\mu}}}
\newcommand{\sss}{\ensuremath{s_{\sigma^2}}}
\newcommand{\sig}{\ensuremath{\sigma^2}}
\newcommand{\hmu}{\ensuremath{\hat{\mu}}}
\newcommand{\hsig}{\ensuremath{\hat{\sigma}^2}}
\newcommand{\hphi}{\ensuremath{\hat{\phi}}}
\newcommand{\hbeta}{\ensuremath{\hat{\beta}}}
\newcommand{\htheta}{\ensuremath{\hat{\theta}}}
\newcommand{\oneon}[1]{\frac{1}{#1}}
\renewcommand{\v}[1]{\mathbf{#1}}
\renewcommand{\P}{\mathbb{P}}
%\newcommand{\E}[1]{\mathbb{E}\left[#1\right]}
\renewcommand{\L}{\mathcal{L}}
\newcommand{\tr}{\,\text{tr}}
\newcommand{\cov}{\ensuremath\text{Cov}}
\newcommand{\var}{\ensuremath\text{Var}}
\renewcommand{\th}{\ensuremath{^{\text{th}}}}
\newcommand\numberthis{\addtocounter{equation}{1}\tag{\theequation}}
\newcommand{\setrange}[2]{\{#1, \ldots, #2\}}
\newcommand{\intrange}[2]{#1, \ldots, #2}
\newcommand{\oneto}[1]{\intrange{1}{#1}}
\newcommand{\class}{\mathcal{C}}
\renewcommand{\symbol}{\mathcal{S}}
\newcommand{\dataset}{\mathcal{D}}
\newcommand{\I}{\mathbb{I}}
\newcommand{\samplespace}{\mathcal{X}}
\newcommand{\logL}{\log \L}
\newcommand{\region}{\mathcal{R}}
\newcommand{\symfamilyk}{\mathcal{F}^{(k)}}
\newcommand{\modelfamily}{\mathcal{G}}

\DeclareMathOperator*{\argmin}{\arg\min}
\DeclareMathOperator*{\argmax}{\arg\max}


\title{SDA with parametric distributions}
\subtitle{Summary of proposed techniques}
\author{Max Fisher}

\begin{document}
\maketitle

\section{Introduction}
Hi my name is Max and here is what I have for show and tell this week.

Suppose that $\mathbf{X} = (X_1, \ldots, X_N)$ are independent and identically distributed
random variables defined on a common sample space $\samplespace$. The true
distribution of the $X_j$ is unknown, but it is to be modelled by a member of a
fixed parametric family $\modelfamily = \{G_\theta\}_{\theta \in \Theta}$, where
each $G_\theta$ admits a probability density or mass function $g(\cdot; \theta)$.

Let $\dataset = (x_1, \ldots, x_N)$ be a realisation of $X_1, \ldots, X_N$.
The set $\dataset$ is termed the microdata. The task is to develop a method of
summarising $\dataset$ to meet the following two goals:
\begin{enumerate}
    \item A high level of information reduction. The summary should use many
        fewer parameters to describe than the number of points in $\dataset$.
    \item Ability to use the only summary to choose a model distribution $G^{*}$
        from $\modelfamily$ that is somehow `good enough' compared to one that
        would have been chosen having full knowledge of $\dataset$ itself.
\end{enumerate}

\section{Method 1: Likelihood integration given observed summary data}

If, using the summary function $h_k: \v{x}_k \mapsto s_k$, each class
$\class_k$ is summarised by the symbol $\symbol_k$ = $(s_k, n_k)$,
then the likelihood function for symbolic data $\v{s} = (s_1, \ldots, s_C)$
is given by
\begin{equation}
    \L^{\symbol}_k(\theta; s_k, n_k)
    = \int_{\region_k} \prod_{i=1}^{n_k} {g(x_{k,i}; \theta)}
    \; d\v{x}_k
    \label{eq:symbolic-class-ll-summary-fn}
\end{equation}
where $\region_k$ denotes integration over the region 
\begin{equation*}
    \region_k = \left\{
        \v{x}_k \in \samplespace^{n_k}: h_k(\v{x}_k) = s_k
            \right\} \subseteq \samplespace^{n_k}.
\end{equation*}
Further, the overall likelihood is
\begin{equation*}
    \L^{\symbol}(\theta; \v{s}, \v{n})
        = \prod_{k=1}^C \L^{\symbol}_k(\theta; s_k, n_k)
\end{equation*}
and the maximum likelihood estimate for the model parameter $\theta$ of
$G_\theta$ is given by
\begin{align}
    \hat{\theta}_{\symbol}
    &= \argmax \, \L^{\symbol}(\theta; \v{s}, \v{n}) \\
    &= \argmax \, \sum_{k=1}^C \ell_k^{\symbol}(\theta; \v{s}, \v{n}).
\end{align}
\subsection{Notes}

\section{Method 2: Minimisation of weighted K-L divergence}
\subsection{Notes}

\section{Method 3: Expected inference on random sample}
\subsection{Notes}

\section{Model fitting/testing}
\begin{itemize}
    \item Need method that does not require homogeneous family of symbols; i.e.
        can fit model to any collection of symbols. This kind of makes the
        `preset' results redundant and focuses more on the computation, which
        makes Methods 2 and 3 seem more applealing, less so Method 1. Method 3
        is kind of dumb/obvious but also very intuitive
    \item Method 1 computation: Do a massive integral
    \item Method 2 computation: find model parameters to minimise weighted sum
        of cross entropies
    \item Method 3 computation [kind of inefficient] Repeated simulations of
        original dataset by iid sampling according to symbols' summary
        distribution and size. For each simulation, compute the MLE same
        distribution as symbols, then compute classical MLE for each sample.
        Return average of MLEs.
\end{itemize}

Two different `schools of thought`:
\begin{enumerate}
    \item{Microdata assumed iid/homogenous across all classes}
    \item{Microdata assumed iid homogenous within classes, not necessarily
        between classes}
\end{enumerate}

\section{Regression}
Can't use OLS with no assumptions, as the whole point is to summarise classes
with parametric distributions.
\begin{align*}
    y_i &= \beta_0 + \beta_1 x_i + \epsilon_i \\
    \epsilon_i &\sim \N(\mu, \sigma^2) \\
    \epsilon_i &= y_i - \beta_0 - \beta_1 x_i
\end{align*}
Issue: don't have $x_i$, $y_i$.

Model classes as either multivariate normal (easy), multivariate t-distribution
(not as easy) or uniform (independent / trivial regression)

\subsection{Testing / simulation setup}
\begin{enumerate}
    \item Define classes - true class distribution + sizes \newline
        \textbf{(Problem: what if class membership is unknown?)}
    \item Sample from class distribution to create microdata
    \item Perform analysis on microdata to get ideal result
    \item Perform symbolic data analysis on microdata to get model fits
\end{enumerate}


\subsection{Multivariate normal assumption}
\begin{equation}
    \left(x_{i, 1}, x_{i, 2}, \ldots, x_{i, k-1}, y_i \right) \sim
    \N(\v{\mu}, \Sigma)
\end{equation}
\begin{equation}
    \v{\mu}
\end{equation}

\subsection{Multivariate t assumption}

\end{document}
